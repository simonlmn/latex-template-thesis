%
% -------------------- LaTeX-Template für Abschlussarbeiten --------------------
%
% Autor:    Simon Lehmann <simon.lehmann@gmx.de>
% Version:  3.0 (2013-04-10)
% 
% Für eine nähere Erklärung des Templates siehe die beiliegende README.md.
%
% --------------------------- Dokumenteneinrichtung ----------------------------
% Einsteigern wird empfohlen direkt zum Abschnitt 'Vorspann' weiter unten zu
% springen und die anderen Einstellungen weiter oben nicht zu verändern.
%
\documentclass[ngerman,a4paper,11pt,twoside,openright,cleardoubleempty,halfparskip]{scrreprt}

% Erforderliche Pakete
\usepackage[utf8]{inputenc}
\usepackage{babel}
\usepackage{graphicx}
\usepackage{fancyhdr}
\usepackage{multirow}
\usepackage{ifthen}
\usepackage{calc}
\usepackage{tabularx}
\usepackage{setspace}

% Laden von weiteren, nützlichen Paketen (je nach Bedarf)
%\usepackage{subfigure} 
%\usepackage{varioref}
%\usepackage{framed}
%\usepackage{longtable}
%\usepackage{floatflt}
%\usepackage{amsmath}
%\usepackage{url}
% ...

% Pakete für Verzeichnisse. Diese sind hier aufgeführt, da mindestens das
% 'glossaries' Paket nach z.B. hyperref geladen werden muss.
\usepackage{bibgerm}
\usepackage[toc,acronym]{glossaries}

% Glossar(e) laden und erstellen
\makeglossaries
\loadglsentries{glossar} % Name der Glossardatei (ohne .tex)

% Befehlsdefinitionen und Einstellungen
%
% Befehle und globale Einstellungen für das LaTeX Template für Abschlussarbeiten
%
% Autor: Simon Lehmann <simon.lmn@gmail.com>
%

% Befehle

% Sorgt dafür, dass Kapitel immer auf einer rechten Seite beginnen.
\newcommand{\clearemptydoublepage}{\newpage{\pagestyle{empty}\cleardoublepage}}

% Befehle zum Setzen und Abrufen der Universität und eines optionalen Logos
\newcommand{\theuniversity}{Hochschule Musterstadt}
\newcommand{\university}[1]{\global\renewcommand{\theuniversity}{#1}}

\newcommand{\theuniversitylogo}{}
\newcommand{\universitylogo}[1]{\global\renewcommand{\theuniversitylogo}{#1}}

% Befehle zum Setzen und Abrufen des Fachbereichs
\newcommand{\thedepartment}{Fachbereich Musterwissenschaften}
\newcommand{\department}[1]{\global\renewcommand{\thedepartment}{#1}}

% Befehle zum Setzen und Abrufen des akad. Grades
\newcommand{\theacademicdegree}{Bachelor of Science (B.Sc.)}
\newcommand{\academicdegree}[1]{\global\renewcommand{\theacademicdegree}{#1}}

% Befehle zum Setzen und Abrufen des Referenten
\newcommand{\thesupervisor}{Prof. Dr. Max Mustermann}
\newcommand{\supervisor}[1]{\global\renewcommand{\thesupervisor}{#1}}

% Befehle zum Setzen und Abrufen des Korreferenten
\newcommand{\thecosupervisor}{Prof. Dr. Max Mustermann}
\newcommand{\cosupervisor}[1]{\global\renewcommand{\thecosupervisor}{#1}}

% Befehle zum Setzen und Abrufen des (optionalen) externen Betreuers
\newcommand{\theexternalsupervisor}{}
\newcommand{\externalsupervisor}[1]{\global\renewcommand{\theexternalsupervisor}{#1}}

% Befehle zum Setzen und Abrufen des Ortsnamen, der in der Erklärung verwendet
% wird
\newcommand{\thecity}{Musterstadt}
\newcommand{\city}[1]{\global\renewcommand{\thecity}{#1}}

\makeatletter % Zugriff auf interne Befehle
% Autorenname, der mit \author{...} gesetzt wurde
\newcommand{\theauthor}{\@author}

% Datum, das mit \date{...} gesetzt wurde
\newcommand{\thedate}{\@date}

% Titel der Arbeit, der mit \title{...} gesetzt wurde
\newcommand{\thetitle}{\@title}

% Befehl um auf leere Inhalte zu testen
\newcommand{\doIfText}[1]{%
	\begingroup
	\sbox0{#1}%
	\ifdim\wd0=\z@
		\endgroup
		\expandafter\@gobble
	\else
		\endgroup
		\expandafter\@firstofone
	\fi}
	
\newcommand{\doIfNoText}[1]{%
	\begingroup
	\sbox0{#1}%
	\ifdim\wd0=\z@
		\endgroup
		\expandafter\@firstofone
	\else
		\endgroup
		\expandafter\@gobble
	\fi}
\makeatother

% Flags für Verbreitungsformen
\newboolean{publishinlibrary}
\setboolean{publishinlibrary}{false}
\newcommand{\publishInLibrary}{\setboolean{publishinlibrary}{true}}
\newcommand{\dontPublishInLibrary}{\setboolean{publishinlibrary}{false}}

\newboolean{publishtitleonline}
\setboolean{publishtitleonline}{false}
\newcommand{\publishTitleOnline}{\setboolean{publishtitleonline}{true}}
\newcommand{\dontPublishTitleOnline}{\setboolean{publishtitleonline}{false}}

\newboolean{publishdocumentonline}
\setboolean{publishdocumentonline}{false}
\newcommand{\publishDocumentOnline}{\setboolean{publishdocumentonline}{true}}
\newcommand{\dontPublishDocumentOnline}{\setboolean{publishdocumentonline}{false}}


% Globale Formatierungseinstellungen
\renewcommand{\encodingdefault}{OT1}
\renewcommand{\familydefault}{cmss} % Schriftfamilie auf Sans Serif
\renewcommand{\glsdisplayfirst}[4]{\textit{#1#4}}
\setcapindent{1em}

% Anpassung der Ränder und Breitenverhältnisse
% Der letzte Wert in oddsidemargin bestimmt die Bindekorrektur
\newcommand{\setbindingcorrection}[1]{%
	\setlength{\oddsidemargin}{2cm - 1in + #1}
	\setlength{\textwidth}{\paperwidth - (1in + \hoffset) - \oddsidemargin - 4cm}
	\setlength{\evensidemargin}{\paperwidth - (1in + \hoffset)*2 - \oddsidemargin - \textwidth}
	\setlength{\marginparwidth}{4cm - \marginparsep - 1cm}
	\setlength{\headwidth}{\textwidth + \marginparsep + \marginparwidth}
}
\setbindingcorrection{0.5cm}

% Kopf- und Fußzeilen einrichten
\pagestyle{fancyplain}
\renewcommand{\chaptermark}[1]{\markboth{#1}{}}
\renewcommand{\sectionmark}[1]{\markright{\thesection\ #1}}
\lhead[\fancyplain{}{\bfseries\thepage}]%
	{\fancyplain{}{\bfseries\rightmark}}
\rhead[\fancyplain{}{\bfseries\leftmark}]%
	{\fancyplain{}{\bfseries\thepage}}
\cfoot{}





% Wenn größerer Zeilenabstand gewünscht ist, entsprechend ändern
%\singlespacing
\onehalfspacing
%\doublespacing

% Bindekorrektur: der Abstand am linken Seitenrand für die Bindung
% ACHTUNG: Dadurch ändert sich die Textbreite, was zur Verschiebung von
% Überschriften, Absätzen und vor allem Abbildungen führen kann.
\setbindingcorrection{0.5cm}

\begin{document}
% ---------------------------------- Vorspann ----------------------------------
% In den folgenden Zeilen werden die wichtigsten Informationen zur Arbeit ge-
% setzt, die dann im Dokument an den entsprechenden Stellen eingefügt werden.
%
\title{Titel der Arbeit}
\author{Name des Autoren}
\date{Datum der Abgabe}
\university{Hochschule RheinMain}
\universitylogo{vorlage/hsrm-logo}
\department{Fachbereich Design Informatik Medien}
\academicdegree{Bachelor of Science (B.Sc.)}
\supervisor{Name des Referenten}
\cosupervisor{Name des Korreferenten}
%\externalsupervisor{Name eines externen Betreuers}
\city{Ort}

% Folgende Befehle steuern, welchen Verbreitungsformen zugestimmt wird. Nicht
% gewünschte Verbreitungsformen entsprechend auskommentieren.
%
\publishInLibrary      % Einstellung der Arbeit in die Bibliothek
\publishTitleOnline    % Veröffentlichung des Titels der Arbeit im Internet
\publishDocumentOnline % Veröffentlichung der Arbeit im Internet

% Vorspann erzeugen, enthält u.a. Titelseite, Erklärungen und Inhaltsverzeichnis
%
% Dokumentenvorspann für das LaTeX Template für Abschlussarbeiten
%
% Enthält die Titelseite, Erklärungen und Inhaltsverzeichnis
%
% Autor: Simon Lehmann <simon.lmn@gmail.com>
%

\pagenumbering{roman} % Bis zum ersten Kapitel mit römischen Seitenzahlen

% Titelseite
%
% Titelseite für das LaTeX Template für Abschlussarbeiten
%
% Autor: Simon Lehmann <simon.lmn@gmail.com>
%

\begin{titlepage}
	\begin{center}
		% Kopf der Seite
		\doIfText{\theuniversitylogo}{\includegraphics{\theuniversitylogo} \\[0.7cm]}
		\doIfNoText{\theuniversitylogo}{{\LARGE\theuniversity} \\[0.7cm]}
		{\thedepartment}
		
		\vfill

		% Mitte der Seite
		{\LARGE Abschlussarbeit} \\[0.5cm]
		{\large zur Erlangung des akademischen Grades} \\[0.5cm]
		{\large \theacademicdegree}
		
		\rule{\textwidth}{1pt}\\[0.5cm]
		{\huge \bfseries \thetitle}\\[0.1cm]
		\rule{\textwidth}{1pt}
		
		\vfill
		
		% Fuß der Seite
		\begin{tabular}{lr}
			Vorgelegt von & \theauthor \\
			am & \thedate \\
			Referent & \thesupervisor \\
			Korreferent & \thecosupervisor
			\doIfText{\theexternalsupervisor}{\\ Externer Betreuer & \theexternalsupervisor}
		\end{tabular}
		
		\vfill
		
	\end{center}
\end{titlepage}
 
\clearemptydoublepage

% Erklärung zur Selbstständigkeit und Verwendung der Arbeit
%
% Erklärungen für das LaTeX Template für Abschlussarbeiten
%
% Enthält eine Erklärungen über die selbstständige Anfertigung der Arbeit
% und wie die Arbeit verbreitet werden darf.
%
% Autor: Simon Lehmann <simon.lmn@gmail.com>
%

\section*{Selbstständigkeitserklärung}

Ich erkläre hiermit,
\begin{itemize}
	\item dass ich die vorliegende Abschlussarbeit selbstständig angefertigt,
	\item keine anderen als die angegebenen Quellen benutzt,
	\item die wörtlich oder dem Inhalt nach aus fremden Arbeiten entnommenen Stellen, bildlichen Darstellungen und dergleichen als solche genau kenntlich gemacht und
	\item keine unerlaubte fremde Hilfe in Anspruch genommen habe.
\end{itemize}

\vspace{3cm}
\makebox[.45\linewidth][l]{\thecity, \thedate}\hfill\rule{.45\linewidth}{0.5pt}\\
\makebox[.45\linewidth][l]{}\hfill\makebox[.45\linewidth][l]{\theauthor}

\vfill

\section*{Erklärung zur Veröffentlichung}

Hiermit erkläre ich mein Einverständnis mit den im folgenden aufgeführten Verbreitungsformen dieser Abschlussarbeit:

\newcolumntype{R}{>{\raggedright\arraybackslash}X}
\begin{tabularx}{\linewidth}{|R|c|c|}
	\hline
	\textbf{Verbreitungsform} & \textbf{Ja} & \textbf{Nein}
	\\\hline
	Einstellung der Arbeit in die Bibliothek der \theuniversity & \ifthenelse{\boolean{publishinlibrary}}{$\times$}{} & \ifthenelse{\boolean{publishinlibrary}}{}{$\times$}
	\\\hline
	Veröffentlichung des Titels der Arbeit im Internet & \ifthenelse{\boolean{publishtitleonline}}{$\times$}{} & \ifthenelse{\boolean{publishtitleonline}}{}{$\times$}
	\\\hline
	Veröffentlichung der Arbeit im Internet & \ifthenelse{\boolean{publishdocumentonline}}{$\times$}{} & \ifthenelse{\boolean{publishdocumentonline}}{}{$\times$}
	\\\hline
\end{tabularx}

\vspace{3cm}
\makebox[.45\linewidth][l]{\thecity, \thedate}\hfill\rule{.45\linewidth}{0.5pt}\\
\makebox[.45\linewidth][l]{}\hfill\makebox[.45\linewidth][l]{\theauthor}

\clearemptydoublepage

% Inhaltsverzeichnis
\tableofcontents 
\clearemptydoublepage

\pagenumbering{arabic}


% --------------------------------- Hauptteil ----------------------------------
% Im diesen Teil werden die einzelnen Kapitel eingefügt, die sinnvollerweise
% im Verzeichnis 'kapitel' abgelegt werden.

\include{kapitel/einleitung}

%\include{kapitel/grundlagen}

% und so weiter ...

% ---------------------------------- Anhänge -----------------------------------
% In diesem Teil werden alle Anhänge eingefügt, die auch als ganz normale Kapi-
% tel abgelegt werden.
\appendix

%\include{kapitel/einanhang}


% Ausgabe des Glossars (oder der Glossare, wenn mehrere definiert sind)
\printglossaries

% Ausgabe des Literaturverzeichnisses
\bibliographystyle{geralpha}
\bibliography{literatur}

\end{document}
