%
% Befehle und globale Einstellungen für das LaTeX Template für Abschlussarbeiten
%
% Autor: Simon Lehmann <simon.lmn@gmail.com>
%

% Befehle

% Sorgt dafür, dass Kapitel immer auf einer rechten Seite beginnen.
\newcommand{\clearemptydoublepage}{\newpage{\pagestyle{empty}\cleardoublepage}}

% Befehle zum Setzen und Abrufen der Universität und eines optionalen Logos
\newcommand{\theuniversity}{Hochschule Musterstadt}
\newcommand{\university}[1]{\global\renewcommand{\theuniversity}{#1}}

\newcommand{\theuniversitylogo}{}
\newcommand{\universitylogo}[1]{\global\renewcommand{\theuniversitylogo}{#1}}

% Befehle zum Setzen und Abrufen des Fachbereichs
\newcommand{\thedepartment}{Fachbereich Musterwissenschaften}
\newcommand{\department}[1]{\global\renewcommand{\thedepartment}{#1}}

% Befehle zum Setzen und Abrufen des akad. Grades
\newcommand{\theacademicdegree}{Bachelor of Science (B.Sc.)}
\newcommand{\academicdegree}[1]{\global\renewcommand{\theacademicdegree}{#1}}

% Befehle zum Setzen und Abrufen des Referenten
\newcommand{\thesupervisor}{Prof. Dr. Max Mustermann}
\newcommand{\supervisor}[1]{\global\renewcommand{\thesupervisor}{#1}}

% Befehle zum Setzen und Abrufen des Korreferenten
\newcommand{\thecosupervisor}{Prof. Dr. Max Mustermann}
\newcommand{\cosupervisor}[1]{\global\renewcommand{\thecosupervisor}{#1}}

% Befehle zum Setzen und Abrufen des (optionalen) externen Betreuers
\newcommand{\theexternalsupervisor}{}
\newcommand{\externalsupervisor}[1]{\global\renewcommand{\theexternalsupervisor}{#1}}

% Befehle zum Setzen und Abrufen des Ortsnamen, der in der Erklärung verwendet
% wird
\newcommand{\thecity}{Musterstadt}
\newcommand{\city}[1]{\global\renewcommand{\thecity}{#1}}

\makeatletter % Zugriff auf interne Befehle
% Autorenname, der mit \author{...} gesetzt wurde
\newcommand{\theauthor}{\@author}

% Datum, das mit \date{...} gesetzt wurde
\newcommand{\thedate}{\@date}

% Titel der Arbeit, der mit \title{...} gesetzt wurde
\newcommand{\thetitle}{\@title}

% Befehl um auf leere Inhalte zu testen
\newcommand{\doIfText}[1]{%
	\begingroup
	\sbox0{#1}%
	\ifdim\wd0=\z@
		\endgroup
		\expandafter\@gobble
	\else
		\endgroup
		\expandafter\@firstofone
	\fi}
	
\newcommand{\doIfNoText}[1]{%
	\begingroup
	\sbox0{#1}%
	\ifdim\wd0=\z@
		\endgroup
		\expandafter\@firstofone
	\else
		\endgroup
		\expandafter\@gobble
	\fi}
\makeatother

% Flags für Verbreitungsformen
\newboolean{publishinlibrary}
\setboolean{publishinlibrary}{false}
\newcommand{\publishInLibrary}{\setboolean{publishinlibrary}{true}}
\newcommand{\dontPublishInLibrary}{\setboolean{publishinlibrary}{false}}

\newboolean{publishtitleonline}
\setboolean{publishtitleonline}{false}
\newcommand{\publishTitleOnline}{\setboolean{publishtitleonline}{true}}
\newcommand{\dontPublishTitleOnline}{\setboolean{publishtitleonline}{false}}

\newboolean{publishdocumentonline}
\setboolean{publishdocumentonline}{false}
\newcommand{\publishDocumentOnline}{\setboolean{publishdocumentonline}{true}}
\newcommand{\dontPublishDocumentOnline}{\setboolean{publishdocumentonline}{false}}


% Globale Formatierungseinstellungen
\renewcommand{\encodingdefault}{OT1}
\renewcommand{\familydefault}{cmss} % Schriftfamilie auf Sans Serif
\renewcommand{\glsdisplayfirst}[4]{\textit{#1#4}}
\setcapindent{1em}

% Anpassung der Ränder und Breitenverhältnisse
% Der letzte Wert in oddsidemargin bestimmt die Bindekorrektur
\newcommand{\setbindingcorrection}[1]{%
	\setlength{\oddsidemargin}{2cm - 1in + #1}
	\setlength{\textwidth}{\paperwidth - (1in + \hoffset) - \oddsidemargin - 4cm}
	\setlength{\evensidemargin}{\paperwidth - (1in + \hoffset)*2 - \oddsidemargin - \textwidth}
	\setlength{\marginparwidth}{4cm - \marginparsep - 1cm}
	\setlength{\headwidth}{\textwidth + \marginparsep + \marginparwidth}
}
\setbindingcorrection{0.5cm}

% Kopf- und Fußzeilen einrichten
\pagestyle{fancyplain}
\renewcommand{\chaptermark}[1]{\markboth{#1}{}}
\renewcommand{\sectionmark}[1]{\markright{\thesection\ #1}}
\lhead[\fancyplain{}{\bfseries\thepage}]%
	{\fancyplain{}{\bfseries\rightmark}}
\rhead[\fancyplain{}{\bfseries\leftmark}]%
	{\fancyplain{}{\bfseries\thepage}}
\cfoot{}



